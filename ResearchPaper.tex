\documentclass[journal]{IEEEtran}
\usepackage{cite}
\usepackage{amsmath,amssymb,amsfonts}
\usepackage{algorithmic}
\usepackage{graphicx}
\usepackage{textcomp}
\usepackage{xcolor}
\def\BibTeX{{\rm B\kern-.05em{\sc i\kern-.025em b}\kern-.08em
    T\kern-.1667em\lower.7ex\hbox{E}\kern-.125emX}}

% My stuff
\newcommand{\textbi}[1]{\textbf{\textit{#1}}}
\usepackage{pifont}% http://ctan.org/pkg/pifont
\newcommand{\cmark}{\ding{51}}
\newcommand{\xmark}{\ding{55}}
% End of my stuff

\begin{document}

\title{Detecting Rogue Switches through Active Verification}

\author{\IEEEauthorblockN{Kevin Kredit}
\IEEEauthorblockA{\textit{GVSU CIS Department} \\
Grand Rapids, Michigan \\
Email: k.kredit.us@ieee.org}
}

\maketitle



\begin{abstract}
Though level 2 switches bear great responsibility in LAN security, detecting rogue switches is
difficult. This paper reviews the threat landscape around switches, overviews rogue switch
prevention measures, and proposes a new double-pronged rogue switch prevention and detection
mechanism. The first prong is TPM-enalbed remote attestation of switches and setting configurations.
The second prong is behavioral verification of switches using generated traffic. The tool leverages
defined switch behavior and the presence of switch security features to detect mis-configured and
potentially malicious switches. Together, the forms of verification support strong assurance as well
as backwards hardware compatibility. Finally, a set of policies designed to create a trusted
network backbone is described.

\iffalse
% original abstract, still interesting for history
Switches have several built-in security features. Most are of the pattern (1) create rule for a
protocol (2) snoop traffic of that protocol (3) drop packets that violate the rule. Examples are
DHCP snooping, dynamic ARP inspection, and MAC limiting. This paper proposes an automated switch
configuration verification tool. This tool would test switches against the defined policies to
ensure that they are properly configured. The test method will be generated traffic specifically
designed to validate a given policy. Depending on the tool mode, this method can detect innocent
errors as well as malicious or rogue switches. Three modes of tool setup of varying convenience and
detection capabilities are investigated: test-bench mode, deployed client-server mode, and deployed
server-only mode.
\fi
\end{abstract}

\begin{IEEEkeywords}
network security, security management, local area networks
\end{IEEEkeywords}



\section{Introduction}

As trusted elements of basic networking infrastructure, Ethernet switches hold great
responsibility. A compromised or maliciously programmed switch poses fearsome threats to network
security. Due to its position between nodes on a LAN, a switch can greatly impact confidentiality,
integrity, and availability of network services. Switches face threats to their inherent
functionality as well as their features designed to secure higher level protocols.

As the threat to a LAN from a compromised switch is so significant, a number of prevention-oriented
mitigations already exist. However, the impact of a compromised switch is great enough to warrant
further research, particularly in how to detect rogue switches.

\subsection{Detecting a Rogue Switch}
How does one detect a disobedient network switch? One method is passive detection, usually by
observing network traffic and parsing it for suspicious behavior. This approach lies solidly within
the realm of intrusion detection systems (IDSs). Machine learning promises a new level of
effectiveness for IDSs, though they is historically only partially effective and by their nature
offer only probabilistic insight.

A second method is proactive verification. This approach could take two forms: self-verification
and behavioral verification. Self-verification means using switch attestation to validate that it
is running trusted firmware as well as booting into an approved configuration. Behavioral
verification means running the switch through a suite of tests and verifying correct behavior.

\subsection{This Paper's Contribution}
This paper begins by summarizing the current state of switch security. It next describes
self-verification, introduces a tool designed to enable behavioral verification, and builds upon the
two to introduce a network management model that uses both techniques to establish a verified
network backbone.



\section{The Switch Security Threat Landscape}
One could classify switch security threats many ways. In this paper, switch security is classified
according to threats inherent in its network position and threats relating to higher level
protocols.

For better or for worse, the switch's role in network security is not limited to switching packets
correctly. As arbiter of LAN access, switches have been tasked with enforcing policies on higher
level protocols that assume particular network behavior. This type of threat is analyzed separately.

\subsection{Inherent Security Threats}
The switch's core job is route packets on a LAN from one level 2 host to another. To perform its
job correctly, it must route packets to the correct destination without modification or delay. This
job can be subverted in three ways, corresponding to the security triad of confidentiality,
integrity, and availability:

\begin{enumerate}
  \item Traffic redirection (confidentiality) % antoher host or internal storage
  \item Traffic modification (integrity)
  \item Denial of service (availability)
\end{enumerate}

Traffic redirection is to route packets to the wrong port, an additional port, or even to internal
storage. Redirection can be used for man-in-the-middle (MITM) or exfiltration attacks. Traffic
modification is to rewrite data as it passes through the switch. This is an MITM attack. Denial of
service is failing to deliver packets at all. This could be done selectively or completely. All
three could be designed to evade IDSs. Redirection could be to internal storage, which would leave
no network trace; modification could be selective, and require comparison of all inbound and
outbound traffic to detect; and denial of service could be selective or strategically timed such
when it is detected, it is too late.

While switch integrity is clearly violated in an arbitrarily programmed malicious switch,
non-malicious switches can also be compromised by malicious hosts. Two ways this is done are
Content-Addressable Memory (CAM) table overflow and MAC address spoofing.

\subsubsection{CAM Overflow} CAM overflow occurs when a switch's limited MAC address to port mapping
memory is exhausted, at which point the switch falls back to behaving like a hub; this is achieved
by spoofing MAC addresses on an interface until the CAM table is full \cite{b1}. CAM overflow is a
confidentiality (misrouted packets) and denial of service (poor network performance) attack.

\subsubsection{MAC Address Spoofing} While MAC address spoofing is an element of the CAM overflow
attack, a MAC address spoof attack is oriented not at overwhelming the switch, but at impersonating
another host's valid MAC address so that the switch updates its CAM table entry and subsequently
forwards future traffic intended for the valid host to the malicious host \cite{b1}. This is a
confidentiality attack that enables MITM.

\subsection{Higher Level Protocols}
Many core networking protocols are designed assuming trust within the LAN. DHCP assumes only one
server (or set of distributed but coordinated servers) exists per LAN. ARP assumes hosts will not
lie about their identity. Due to mis-configuration and malicious behavior, such assumptions are
frequently wrong. Because of the switch's role as gatekeeper between a host and the LAN, it is a
convenient place to implement defenses against attacks on these higher level protocols.

% TODO finalize scope of this paper
This paper focuses not on how to configure the switch security features that defend DHCP and ARP,
but on rogue switch security implications and detection. The threat considered here is not ``rogue
DHCP'' or ``ARP spoofing,'' but that of a compromised switch which uses misapplications of these
security features to enable further attacks. For example, a compromised switch could block the valid
DHCP server and enable a rogue DHCP server.

In order to understand the impacts of a rogue switch, therefore, one must understand their high
level protocol security features. In addition to their importance to rogue switch impact, these
features provide a possible means for rogue switch detection.

Two protocols that switches protect are DHCP and ARP. Both have clear impacts on confidentiality and
availability. DHCP attacks impact integrity if the rogue DHCP server sends wrong router and
default gateway addresses. ARP attacks impact integrity if they are used to perform MITM attacks.

\subsubsection{DHCP Snooping} DHCP is susceptible to spoofing and starvation; DHCP snooping is a
switch security feature that enables DHCP offer filtering and generation of a DHCP binding table
used to validate incoming packets and limit the number of IP addresses on a port \cite{b2}.

\subsubsection{Dynamic ARP Inspection} ARP is used by hosts to tie IP addresses to MAC addresses.
ARP is susceptible to spoofing and ``poisoning''; dynamic ARP inspection (DAI) is a switch security
feature that uses the DHCP snooping binding table to validate ARP messages before forwarding them
\cite{b2}.

\subsection{Threat Summary}
The switch security threats described in the previous sections are summarized here, organized
according to the CIA triad:
\begin{enumerate}
  \item Confidentiality
  \begin{enumerate}
    \item misrouted packets
    \item internally stored packets
    \item fall back to hub (all packets broadcast)
    \item compromised DHCP defense (rogue DHCP)
    \item compromised ARP defense (MITM)
  \end{enumerate}
  \item Integrity
  \begin{enumerate}
    \item traffic modification
    \item incorrect MAC to port mapping, enabling MITM
    \item compromised DHCP defense (rogue DHCP)
    \item compromised ARP defense (MITM)
  \end{enumerate}
  \item Availability
  \begin{enumerate}
    \item denial of service
    \item fall back to hub (decreased performance)
    \item compromised DHCP defense (exhausted DHCP)
    \item compromised ARP defense (poisoned tables)
  \end{enumerate}
\end{enumerate}

\subsection{Threat Vectors}
Given the high impacts of rogue switches, the pressing question is ``how could a rogue switch be
introduced to the network?'' The possibilities are as follows:
\begin{enumerate}
  \item A valid switch could be compromised via
  \begin{enumerate}
    \item a host sending malicious traffic (e.g., MAC spoofing)
    \item stolen (or guessed) management credentials
    \item software vulnerability
    \item physical modification
  \end{enumerate}
  \item A malicious switch could be inserted to the network at
  \begin{enumerate}
    \item a critical network location
    \item a public facing port
  \end{enumerate}
\end{enumerate}

\iffalse
% TODO finish the table? questionable value
Not all attacks are enabled by all vectors. Table \ref{table:vectors} summarizes their
applicability. While many attacks are possible, checkmarks (\cmark) designate stronger dangers and
x-marks (\xmark) designate lesser dangers.
\begin{table}[htbp]
  \caption{Switch Attacks and Threat Vectors}
  \begin{center}
    \begin{tabular}{|l|c|c|c|c|c|c|}
      \hline
      \multicolumn{1}{|c}{\textbf{Switch}} & \multicolumn{6}{|c|}{\textbf{Threat Vector}} \\
                                                                                        \cline{2-7}
      \multicolumn{1}{|c|}{\textbf{Threat}} & \textbi{1a} & \textbi{1b} & \textbi{1c} & \textbi{1d}
                                                              & \textbi{2a} & \textbi{2b} \\ \hline
      1(a) &        &        & \cmark & \cmark & \cmark & \cmark \\ \hline
      1(b) & \xmark & \xmark & \cmark & \cmark & \cmark & \cmark \\ \hline
      1(c) &        & \cmark & \cmark & \cmark & \cmark & \cmark \\ \hline
      1(d) & \cmark & \xmark & \cmark & \cmark & \cmark & \cmark \\ \hline
      1(e) & \cmark & \xmark & \cmark & \cmark & \cmark & \cmark \\ \hline
      2(a) & \xmark & \cmark & \cmark & \cmark & \cmark & \cmark \\ \hline
      2(b) & \xmark & \cmark & \cmark & \cmark & \cmark & \cmark \\ \hline
      2(c) &        & \cmark & \cmark & \cmark & \cmark & \cmark \\ \hline
      2(d) & \cmark & \xmark & \cmark & \cmark & \cmark & \cmark \\ \hline
      3(a) & \xmark & \cmark & \cmark & \cmark & \cmark & \cmark \\ \hline
      3(b) & \xmark & \cmark & \cmark & \cmark & \cmark & \cmark \\ \hline
    \end{tabular}
    \label{table:vectors}
  \end{center}
\end{table}
\fi


\iffalse
% TODO finish the table? questionable value
% TODO is this table "risk" or "criticality"?
% TODO determine TBDs
% TODO introduce table and describe assumptions (e.g. unencrypted traffic)
\begin{table}[htbp]
\caption{Level 2 Security and the InfoSec Triad}
\begin{center}
\begin{tabular}{|c|c|c|c|}
  \hline
  \textbf{Level 2} & \multicolumn{3}{|c|}{\textbf{Triad Impact}} \\ \cline{2-4}
  \textbf{Feature} & \textbi{Confidentiality}& \textbi{Integrity}& \textbi{Availability} \\ \hline

  DHCP Snooping  & None   & Medium & Medium \\ \hline
  ARP inspection & Medium & Medium & High   \\ \hline
  MAC limiting   & None   & Medium & High   \\ \hline
  Port security  & TBD    & TBD    & TBD    \\ \hline
\end{tabular}
\label{tab1}
\end{center}
\end{table}
\fi



\section{Current Mitigations}

\subsection{Basics}
No new security invention replaces the need for strong fundamentals. A few necessary first steps
are presented before this paper's contribution is considered.

\subsubsection{Password management}
Changing switches' administrative credentials from the default is the first step in securing the
level 2 network. Changing the credentials is more important than enforcing physical security
because failing to do so enables effortless, automated remote compromise. Even if the organization
does not intend to implement switches' available security features, changing the default
credentials is necessary to prevent an attacker from configuring the switch to his interests. This
mitigates threat vector 1(b), stolen management credentials.

\subsubsection{Physical security}
Preventing physical hardware access prevents bad actors from simply installing their hardware on
your network. Particularly for backbone portions of the network, physical access must be controlled.
This mitigates threat vectors 1(d), physical modification, and 2(a), switch insertion at a
critical network location.

\subsubsection{Patching and other fundamentals}
Implementing a patching program, appropriate network segmentation, and intrusion prevention systems
helps keep bad actors out of the network in the first place. To the extent switches and routers get
software patches, they should be applied in a timely manner. This mitigates threat vectors 1(a), a
host sending malicious traffic, and 1(c), a software vulnerability in the networking equipment.

\subsubsection{Implementing available security features}
After the previous steps have been taken, implementing the switches' available security features is
the next logical step, and indeed is a prerequisite the more capable versions of the active
verification approach. MAC limiting, DHCP snooping, DAI, and port security mitigate threat vectors
1(a), a host sending malicious traffic, and 2(b), switch insertion at public facing ports \cite{b1}.


\subsection{Encryption and Higher Level Protocols}
When mitigating security risk, besides decreasing likelihood, decreasing the cost of successful
attacks is another valid approach. Using encryption in higher level protocols such as IPsec, TLS,
SSH, and SFTP, or even using level 2 encryption \cite{b3} nearly eliminates a rogue switch's
capability to violate confidentiality or integrity.


\subsection{IEEE 802.11i Wi-Fi Security}
At first blush one may suggest implementing Wi-Fi 802.11i security on the wired Ethernet LAN. After
all, there exists more research on the subject \cite{b4}\cite{b5}. However, Wi-Fi security measures
are not well suited to wired LANs. The main reason is practical: the overhead is significant and
the threat likelihood is much less on a LAN. Wireless networks can be passively observed by anyone
within range, whereas Ethernet LAN attacks require either a remote exploit or physical access.

Additionally, wireless and wired networks are inherently structured differently: wireless networks
are broadcast traffic routed through the access point, whereas wired networks support point to
point communication and have switches that arbitrate access.
% TODO keep? feels like it needs to be expanded upon, or is completely unnecessary for a smarter
% TODO audience...



\section{Switch Self-authentication}
The preferred way to detect and prevent rogue switches from joining a network is to place the
verification burden on the switch.

\subsection{Self-authentication with a TPM}
Using a Trusted Platform Module would enable the network to
verify that switches are operating trusted software stacks on trusted hardware. Such attestation
methods for verifying client integrity are well understood \cite{b6}. In particular, when a
TPM-enabled device boots, the TPM generates a certification signed with its attestation identity
key and a certificate from a trusted certification authority that was installed at time of
manufacture; this certificate is sent to challengers during attestation; challengers verify the
signatures and assign trust to the client \cite{b7}.

% TODO make a flow chart instead. Also, what if it is _not_ connected to a managed switch that
% TODO is already part of the verified network? Then beyond the core trusted network; uncontrolled
The basic authentication flow would be as follows:
\begin{enumerate}
  \item A unknown switch joins the network. It is connected to a managed switch that is already
  part of the verified network.
  \item The existing switch challenges the new switch's identity.
  \item The unknown switch either:
  \begin{enumerate}
    \item Responds with a trusted attestation key. It joins the network as a verified element.
    \item Responds with an untrusted attestation key. It is treated according to the network's
    security policy.
    \item Is unequipped for the identity challenge. It is treated according to the network's
    security policy.
  \end{enumerate}
\end{enumerate}

This mitigates threat vectors 2(a) and 2(b), switch insertion. For the case of verified switches
becoming compromised after initial attestation, periodic re-attestation of their configuration is
required. If a switch is found to have been compromised, it is treated according to the network's
security policy. This mitigates the remaining threat vectors.

The ``network security policies'' relating to how to treat untrusted switches are discussed in
section \ref{sec:policies}.

\subsection{Self-authentication Without a TPM}
Self-authentication can be achieved without a TPM if switches are pre-configured with a shared
secret. While this allows self-authentication with current-generation hardware, it is vulnerable
to spoofing if the secret is compromised and it is less automated as it requires manual
configuration.



\section{The Behavioral Verification Tool}
Self-authentication is the preferred means of switch validation. However, switches typically are
outfitted with TPMs. Even if that were to begin, organizations would rightly not be able to justify
the cost of updating the millions of deployed switches worldwide for the sake of this feature. In
order to build a verified network backbone, there needs to be a way to validate currently deployed
switches.

This paper proposes a behavioral verification tool that establishes trust in switches by putting
them through thorough tests. Such test-based trust is not as strong as attestation. All cases could
not be tested; for example, a malicious switch might only filter packets from certain source MAC
addresses. The tests could be evaded; for example, the device may detect that it is under test, and
perform as expected until the test is over. These possibilities are of little concern for two
reasons. First, both such cases require not just a maliciously \textit{configured} switch, but a
maliciously \textit{programmed} switch. This is a less likely threat, and can be partially
addressed through the switch security policies. Second, as with attestation, tests would be
repeated periodically, making test detection non-trivial.

The behavior verification tool works by functionally testing corner cases as well as random cases
of each switch configuration policy. If a switch is correctly configured and acts as expected, it
passes the tests. If a switch is mis-configured or acts in any unexpected way, it fails. Passing
switches join the network as a verified element. Failing switches are treated according to the
network's security policy.

The proposed tool has three modes that offer increasing thoroughness at the cost of increasing
configuration and labor.

\subsection{Deployed Switch Server-only Mode}
This mode of the tool only requires only a server. It is called ``deployed switch'' because the
steps can be performed on switches that are already deployed and new switches require no special
steps prior to deployment. The server has knowledge of
the network topology and generates traffic in such a manner that tests the policies of the new
switch. Server-only mode is limited in what traffic it can send. For example, if it sends traffic
that violates security policies, it will be rejected on the way to the target switch.
Server-only mode still allows for automatic switch configuration, though: when a new switch
joins the network, the server detects the make and model, logs in, and sends the appropriate
configuration commands.
\subsubsection{Advantages}
\begin{itemize}
	\item Low configuration overhead
	\item Can automatically configure new switches
\end{itemize}
\subsubsection{Disadvantages}
\begin{itemize}
	\item Very limited live-traffic behavioral verification
\end{itemize}

\subsection{Deployed Switch Client-server Mode}
This mode of the tool adds the concept of clients that assist the server in generating and detecting
policy-testing traffic. Suitable hosts for the client application include endpoints and managed
switches themselves. The server coordinates the clients to generate a as thorough a set of
test traffic as possible. Because of the additional nodes available to the tool, possibly including
devices
attached directly to the switch, the properties it can test in this mode are more thorough. The
server still logs in to perform administrative functions on managed switches.
\subsubsection{Advantages}
\begin{itemize}
	\item Can automatically configure new switches
	\item More capable live-traffic behavioral verification
\end{itemize}
\subsubsection{Disadvantages}
\begin{itemize}
	\item High configuration overhead
\end{itemize}
% TODO what protocols can we use to test things? ICMP, Echo, SNMP?

\subsection{Test-bench Mode}
This mode of the tool requires only a server, but requires that the switch be taken out of
deployment or tested before deployment. By sequentially testing each port of the switch with
arbitrary traffic, the server is
able to fully verify all the behavior expected of the switch. It begins by logging in to perform
administrative functions on managed switches, then runs through the test suite.



\section{Building a Verified Network Backbone}

\subsection{Switch Addition}
By establishing trust in switches using self-authentication and behavioral verification, network
administrators can establish a core or ``backbone'' of trusted switches. This backbone grows
dynamically in accordance with the flow chart shown in Figure \ref{fig:switch-addition}.

\begin{figure}[htbp]
\centerline{\includegraphics[width=0.49\textwidth]{images/switch-addition}}
\caption{The process of dynamically growing the trusted network backbone.}
\label{fig:switch-addition}
\end{figure}

\subsection{Security Policies}\label{sec:policies}
Connections between trusted switches can
be configured permissively, while connections between trusted and untrusted switches can be
configured with flexible security policies determining how to handle connections with new devices.
Example policies range from strict to permissive:
\begin{itemize}
  \item Strict: until a new switch is successfully verified (attested or tested),
  \begin{itemize}
    \item deny all outbound non-attestation or non-test related traffic on the port
    \item deny all inbound non-attestation, non-test, and non-configuration related traffic on the
    port
  \end{itemize}
  \item Limited: until a new switch has successfully verified (attested or tested), or if a switch
  is unable to verify,
  \begin{itemize}
    \item treat the port as untrusted
    \item set a limit on the number of hosts allowed to be tied to the port
  \end{itemize}
  \item Permissive: regardless of verification, treat the port like normal
\end{itemize}

Policies can vary depending on attestation or test method. For example, remotely-attesting switches
could be considered fully trusted, thoroughly tested switches could be mostly trusted, partially
tested switches could be less trusted, and untested switches could be untrusted. The level of trust
in a switch translates into port security features applied to that switch's connection to the rest
of the network, an example being the number of hosts allowed to connect through it.

%\subsection{Applicability}
%Not all switches have the security features on which the proposed tool relies.
%Need a ``core network'' perimeter. (But we want to move away from perimeters :( )


\section{Conclusion}
Rogue and compromised switches represent grave threats to LAN security. Though relatively low in
likelihood, preventing and detecting rogue switches is a valuable step to take. After taking the
necessary basic steps in securing the LAN and using encrypted communications when possible,
switch validation allows the creation of verified LAN backbones. TPM-based remote attestation
enables strong verification, while behavior-based verification enables backwards compatibility with
currently deployed equipment. Both measures mitigate the threat of rogue switches through active
prevention and detection.


% \begin{equation}
% a+b=\gamma\label{eq}
% \end{equation}

% \paragraph{A labeled paragraph} With some text.

% \begin{table}[htbp]
% \caption{Table Type Styles}
% \begin{center}
% \begin{tabular}{|c|c|c|c|}
% \hline
% \textbf{Table}&\multicolumn{3}{|c|}{\textbf{Table Column Head}} \\
% \cline{2-4} 
% \textbf{Head} & \textbf{\textit{Table column subhead}}& \textbf{\textit{Subhead}}& 
%\textbf{\textit{Subhead}} \\
% \hline
% copy& More table copy$^{\mathrm{a}}$& &  \\
% \hline
% \multicolumn{4}{l}{$^{\mathrm{a}}$Sample of a Table footnote.}
% \end{tabular}
% \label{tab1}
% \end{center}
% \end{table}

% \begin{figure}[htbp]
% \centerline{\includegraphics{fig1.png}}
% \caption{Example of a figure caption.}
% \label{fig}
% \end{figure}


\section*{Acknowledgment}
Kevin Kredit thanks Professor Vijay Bhuse and his CIS 616 ``Data Security and Privacy'' Winter 2019
classmates for their constructive input over the course of the semester.


\begin{thebibliography}{00}

\bibitem{b1} ``Layer 2 Security Features on Cisco Catalyst Layer 3 Fixed Configuration Switches
Configuration Example'', Cisco, 2007. [Online]. Available:
https://www.cisco.com/c/en/us/support/docs/switches/catalyst-3750-series-switches/72846-layer2-secftrs-catl3fixed.html.
 [Accessed: 05- Apr- 2019].

\bibitem{b2} Y. Bhaiji, ``Security Features on Switches $>$ Securing Layer 2'', Ciscopress.com,
2008. [Online]. Available: http://www.ciscopress.com/articles/article.asp?p=1181682. [Accessed: 05-
Apr- 2019].

\bibitem{b3} ``Senetas CN Series Layer 2 Encryption solutions'', Senetas. [Online]. Available:
https://www.senetas.com/products/cn-encryptors/. [Accessed: 05- Apr- 2019].

\bibitem{b4} R. Beyah, S. Kangude, G. Yu, B. Strickland and J. Copeland, ``Rogue access point
detection using temporal traffic characteristics,'' \textit{IEEE Global Telecommunications
Conference, 2004. GLOBECOM '04.}, Dallas, TX, 2004, pp. 2271-2275 Vol.4. doi:
10.1109/GLOCOM.2004.1378413

\bibitem{b5} L. Watkins, R. Beyah and C. Corbett, ``A passive approach to rogue access point
detection,''  \textit{IEEE GLOBECOM 2007 - IEEE Global Telecommunications Conference}, Washington,
DC, 2007, pp. 355-360. doi: 10.1109/GLOCOM.2007.73

\bibitem{b6} R. Sailer, T. Jaeger, X. Zhang, and L. van Doorn, ``Attestation-based policy
enforcement for remote access,'' \textit{The 11th ACM conference on Computer and communications
security (CCS '04)}, ACM, New York, NY, USA, 2004, pp. 308-317. doi: 10.1145/1030083.1030125

\bibitem{b7} T. Garfinkel, M. Rosenblum and D. Boneh, ``Flexible OS Support and Applications for
Trusted Computing'', in \textit{HotOS}, 2003, pp. 145-150.

\end{thebibliography}


\end{document}
