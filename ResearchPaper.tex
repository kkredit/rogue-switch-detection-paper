\documentclass[journal]{IEEEtran}
\usepackage{cite}
\usepackage{amsmath,amssymb,amsfonts}
\usepackage{algorithmic}
\usepackage{graphicx}
\usepackage{textcomp}
\usepackage{xcolor}
\def\BibTeX{{\rm B\kern-.05em{\sc i\kern-.025em b}\kern-.08em
    T\kern-.1667em\lower.7ex\hbox{E}\kern-.125emX}}
\newcommand{\textbi}[1]{\textbf{\textit{#1}}}
\begin{document}

\title{Detecting Rogue Switches through Active Verification}

\author{\IEEEauthorblockN{Kevin Kredit}
\IEEEauthorblockA{\textit{GVSU CIS Department} \\
Grand Rapids, Michigan \\
Email: k.kredit.us@ieee.org}
}

\maketitle

\begin{abstract}
Switches have several built-in security features. Most are of the pattern (1) create rule for a
protocol (2) snoop traffic of that protocol (3) drop packets that violate the rule. Examples are
DHCP snooping, dynamic ARP inspection, and MAC limiting. This paper proposes a automated switch
configuration verification tool. This tool would test switches against the defined policies to
ensure that they are properly configured. The test method will be generated traffic specifically
designed to validate a given policy. Depending on the tool mode, this method can detect innocent
errors as well as malicious or rogue switches. Three modes of tool setup of varying convenience and
detection capabilities are investigated: test-bench mode, deployed client-server mode, and deployed
server-only mode.
% TODO update with results. And make sure it represents the paper as a whole.
\end{abstract}

\begin{IEEEkeywords}
network security, security management, local area networks
\end{IEEEkeywords}

\section{Introduction}

As trusted elements of basic networking infrastructure, Ethernet switches hold great
responsibility. A compromised or maliciously programmed switch poses fearsome threats to network
security.
% TODO add some more details here after elaborate in Threats section

How does one detect a disobedient network
switch?

The initial inspiration

\section{Threat Vectors}
% TODO "Vectors" or "Models"?

\subsection{Nullifying Switch Security Policies}
This section focuses on threats from a compromised switch; i.e., threats introduced by nullifying or
maliciously influencing switch security feature configurations.

\begin{itemize}
  \item Disabling DHCP Snooping
  \item Disabling ARP inspection
  \item Disabling MAC limiting
  \item Disabling Port security
  \item Maliciously configuring DHCP Snooping
  \item Maliciously configuring ARP inspection
  \item Maliciously configuring MAC limiting
  \item Maliciously configuring Port security
\end{itemize}

% TODO is this table "risk" or "criticality"?
% TODO determine TBDs
% TODO introduce table and describe assumptions (e.g. unencrypted traffic)
\begin{table}[htbp]
\caption{Level 2 Security and the InfoSec Triad}
\begin{center}
\begin{tabular}{|c|c|c|c|}
  \hline
  \textbf{Level 2} & \multicolumn{3}{|c|}{\textbf{Triad Impact}} \\ \cline{2-4}
  \textbf{Feature} & \textbi{Confidentiality}& \textbi{Integrity}& \textbi{Availability} \\ \hline

  DHCP Snooping  & None   & Medium & Medium \\ \hline
  ARP inspection & Medium & Medium & High   \\ \hline
  MAC limiting   & None   & Medium & High   \\ \hline
  Port security  & TBD    & TBD    & TBD    \\ \hline
\end{tabular}
\label{tab1}
\end{center}
\end{table}

\subsection{Leveraging the Switch's Visibility}
This section focuses on threats from a malicious switch; i.e., threats enabled by an arbitrarily
programmed switch.

\begin{itemize}
  \item Traffic inspection
  \item Traffic redirection
  \item Traffic modification
  \item Denial of service
\end{itemize}

\section{Mitigations}

\subsection{Basics}
No clever security invention replaces the need for strong fundamentals. A few necessary first steps
are presented before this paper's contribution is considered.

\subsubsection{Password management}
Changing the switches' administrative credentials from the default remains the most important first
step in securing the level 2 network. Changing the credentials is more important than physical
security because failing to do so enables effortless, automated remote compromise. Even if the
organization does not intend to implement the available security features, changing the default
credentials is necessary to prevent an attacker from configuring the switch to his interests.
% TODO protects against vector XYZ

\subsubsection{Physical security}
Preventing physical hardware access prevents bad actors from simply installing their hardware on
your network. Particularly for backbone portions of the network, physical access must be controlled.
% TODO protects against vector XYZ

\subsubsection{Implementing Available Security Features}
After passwords are changed and physical access is locked down, implementing the available security
features offered is the next logical step, and indeed is a prerequisite for the active verification
approach.
% TODO protects against vector XYZ

\subsection{Layer 2 Encryption}
% TODO e.g. https://www.senetas.com/products/cn-encryptors/ or
%https://www.thalesesecurity.com/solutions/use-case/data-security-and-encryption/layer-2-encryption

\subsection{Inherent Differences to Wi-Fi Security}
At first blush one may suggest implementing Wi-Fi 802.11i security on the wired Ethernet LAN. The
main reason this is not done is practical: the overhead is significant and the threat likelihood
is much less. Wireless networks can be passively observed

\subsection{Active Switch Verification}


\section{Active Verification Tool Overview}


\section{Active Verification Tool Mode Analysis}

\subsection{Test-bench Mode}

\subsection{Deployed Client-server Mode}

\subsection{Deployed Server-only Mode}


\section{Active Verification Tool Limitations}

\subsection{Applicability}
Not all switches have the security features on which the proposed tool relies.
Need a "core network" perimeter. (But we want to move away from perimeters :( )


\section{Conclusion}


% \section{A Section}\label{section-label}

% \subsection{A Subsection}

% A citation \cite{b1}.

% \textbf{Some boldface.}

% \textit{Some italics.}

% \begin{itemize}
% \item A list item
% \end{itemize}

% \begin{equation}
% a+b=\gamma\label{eq}
% \end{equation}

% \paragraph{A labeled paragraph} With some text.

% \begin{table}[htbp]
% \caption{Table Type Styles}
% \begin{center}
% \begin{tabular}{|c|c|c|c|}
% \hline
% \textbf{Table}&\multicolumn{3}{|c|}{\textbf{Table Column Head}} \\
% \cline{2-4} 
% \textbf{Head} & \textbf{\textit{Table column subhead}}& \textbf{\textit{Subhead}}& 
%\textbf{\textit{Subhead}} \\
% \hline
% copy& More table copy$^{\mathrm{a}}$& &  \\
% \hline
% \multicolumn{4}{l}{$^{\mathrm{a}}$Sample of a Table footnote.}
% \end{tabular}
% \label{tab1}
% \end{center}
% \end{table}

% \begin{figure}[htbp]
% \centerline{\includegraphics{fig1.png}}
% \caption{Example of a figure caption.}
% \label{fig}
% \end{figure}


\section*{Acknowledgment}
Kevin Kredit thanks Professor Vijay Bhuse and his CIS 616 ``Data Security and Privacy'' Winter '19
classmates for their constructive input over the course of the semester.


\begin{thebibliography}{00}
\bibitem{b1} G. Eason, B. Noble, and I. N. Sneddon, ``On certain integrals of Lipschitz-Hankel type
involving products of Bessel functions,'' Phil. Trans. Roy. Soc. London, vol. A247, pp. 529--551,
April 1955.
\end{thebibliography}


\end{document}
